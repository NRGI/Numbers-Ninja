\documentclass[]{article}
\usepackage{lmodern}
\usepackage{amssymb,amsmath}
\usepackage{ifxetex,ifluatex}
\usepackage{fixltx2e} % provides \textsubscript
\ifnum 0\ifxetex 1\fi\ifluatex 1\fi=0 % if pdftex
  \usepackage[T1]{fontenc}
  \usepackage[utf8]{inputenc}
\else % if luatex or xelatex
  \ifxetex
    \usepackage{mathspec}
  \else
    \usepackage{fontspec}
  \fi
  \defaultfontfeatures{Ligatures=TeX,Scale=MatchLowercase}
\fi
% use upquote if available, for straight quotes in verbatim environments
\IfFileExists{upquote.sty}{\usepackage{upquote}}{}
% use microtype if available
\IfFileExists{microtype.sty}{%
\usepackage{microtype}
\UseMicrotypeSet[protrusion]{basicmath} % disable protrusion for tt fonts
}{}
\usepackage[margin=1in]{geometry}
\usepackage{hyperref}
\hypersetup{unicode=true,
            pdftitle={Numbers Ninja: Class 1},
            pdfborder={0 0 0},
            breaklinks=true}
\urlstyle{same}  % don't use monospace font for urls
\usepackage{graphicx,grffile}
\makeatletter
\def\maxwidth{\ifdim\Gin@nat@width>\linewidth\linewidth\else\Gin@nat@width\fi}
\def\maxheight{\ifdim\Gin@nat@height>\textheight\textheight\else\Gin@nat@height\fi}
\makeatother
% Scale images if necessary, so that they will not overflow the page
% margins by default, and it is still possible to overwrite the defaults
% using explicit options in \includegraphics[width, height, ...]{}
\setkeys{Gin}{width=\maxwidth,height=\maxheight,keepaspectratio}
\IfFileExists{parskip.sty}{%
\usepackage{parskip}
}{% else
\setlength{\parindent}{0pt}
\setlength{\parskip}{6pt plus 2pt minus 1pt}
}
\setlength{\emergencystretch}{3em}  % prevent overfull lines
\providecommand{\tightlist}{%
  \setlength{\itemsep}{0pt}\setlength{\parskip}{0pt}}
\setcounter{secnumdepth}{0}
% Redefines (sub)paragraphs to behave more like sections
\ifx\paragraph\undefined\else
\let\oldparagraph\paragraph
\renewcommand{\paragraph}[1]{\oldparagraph{#1}\mbox{}}
\fi
\ifx\subparagraph\undefined\else
\let\oldsubparagraph\subparagraph
\renewcommand{\subparagraph}[1]{\oldsubparagraph{#1}\mbox{}}
\fi

%%% Use protect on footnotes to avoid problems with footnotes in titles
\let\rmarkdownfootnote\footnote%
\def\footnote{\protect\rmarkdownfootnote}

%%% Change title format to be more compact
\usepackage{titling}

% Create subtitle command for use in maketitle
\newcommand{\subtitle}[1]{
  \posttitle{
    \begin{center}\large#1\end{center}
    }
}

\setlength{\droptitle}{-2em}

  \title{Numbers Ninja: Class 1}
    \pretitle{\vspace{\droptitle}\centering\huge}
  \posttitle{\par}
    \author{Hari Subhash\\
Data Scientist @NRGI}
    \preauthor{\centering\large\emph}
  \postauthor{\par}
      \predate{\centering\large\emph}
  \postdate{\par}
    \date{2018-09-13}


\begin{document}
\maketitle

\subsection{Pre-requisites}\label{pre-requisites}

Before we get started with the lesson, there are a few things we need to
install on our local machine.

\begin{enumerate}
\def\labelenumi{\arabic{enumi}.}
\tightlist
\item
  \href{https://cran.r-project.org/mirrors.html}{Install R}: Any of the
  links on that page should work
\item
  \href{https://www.rstudio.com/products/rstudio/download/}{Install
  Rstudio}: Choose the option that is suited for your Operating System
  (OS).
\end{enumerate}

\subsection{Get familiar with your working
environment}\label{get-familiar-with-your-working-environment}

While R ships with a basic code editor it does not provide any
additional functionality to make our lives easier. This is where R
Studio comes in. R Studio is an integrated development environment (IDE)
that is dedicated to code in R. The basic version is free and by far the
most popular and easy to install code editor amongst R users. Lets look
at its various components

\paragraph{What are the different parts of the Rstudio
IDE?}\label{what-are-the-different-parts-of-the-rstudio-ide}

\paragraph{What are the different types of files you can use to write R
code?}\label{what-are-the-different-types-of-files-you-can-use-to-write-r-code}

\textbf{⚡Ninja Tasks⚡}

\begin{enumerate}
\def\labelenumi{\arabic{enumi}.}
\tightlist
\item
  Create a Numbers Ninja Folder on your local machine
\item
  Create an R Notebook and save it in this folder
\item
  Install the following package - tidyverse, wbstats and nycflights13
  using Rstudio
\end{enumerate}

\subsection{A quick markdown detour}\label{a-quick-markdown-detour}

\paragraph{What is a markup language?}\label{what-is-a-markup-language}

\begin{quote}
In computer text processing, a markup language is a system for
annotating a document in a way that is syntactically distinguishable
from the text. The idea and terminology evolved from the ``marking up''
of paper manuscripts, i.e., the revision instructions by editors,
traditionally written with a blue pencil on authors' manuscripts. In
digital media, this ``blue pencil instruction text'' was replaced by
tags, that is, instructions are expressed directly by tags or
``instruction text encapsulated by tags.'' However the whole idea of a
mark up language is to avoid the formatting work for the text, as the
tags in the mark up language serve the purpose to format the appropriate
text (like a header or beginning of a next para\ldots{}etc.). Every tag
used in a Markup language has a property to format the text we write. -
Wikipedia
\end{quote}

Some examples of markup languages includes -
\href{https://en.wikipedia.org/wiki/TeX}{TeX},
\href{https://en.wikipedia.org/wiki/LaTeX}{LaTeX},
\href{https://en.wikipedia.org/wiki/HTML}{HTML} etc. TeX markup language
that created a
\href{https://en.wikipedia.org/wiki/Typesetting}{typesetting system} in
computers for high quality books. LaTeX is used extensively in academia
to create documents. HTML is the language used to create and organize
content on the web. All websites use HTML.

\paragraph{What is R Markdown?}\label{what-is-r-markdown}

\begin{quote}
The document format ``R Markdown'' was first introduced in the knitr
package (Xie 2015, 2018d) in early 2012. The idea was to embed code
chunks (of R or other languages) in Markdown documents. In fact, knitr
supported several authoring languages from the beginning in addition to
Markdown, including LaTeX, HTML, AsciiDoc, reStructuredText, and
Textile. Looking back over the five years, it seems to be fair to say
that Markdown has become the most popular document format, which is what
we expected. The simplicity of Markdown clearly stands out among these
document formats.
\end{quote}

\begin{quote}
However, the original version of Markdown invented by John Gruber was
often found overly simple and not suitable to write highly technical
documents. For example, there was no syntax for tables, footnotes, math
expressions, or citations. Fortunately, John MacFarlane created a
wonderful package named Pandoc (\url{http://pandoc.org}) to convert
Markdown documents (and many other types of documents) to a large
variety of output formats. More importantly, the Markdown syntax was
significantly enriched. Now we can write more types of elements with
Markdown while still enjoying its simplicity.
\end{quote}

\begin{quote}
In a nutshell, R Markdown stands on the shoulders of knitr and Pandoc.
The former executes the computer code embedded in Markdown, and converts
R Markdown to Markdown. The latter renders Markdown to the output format
you want (such as PDF, HTML, Word, and so on). -
\href{https://bookdown.org/yihui/rmarkdown/}{R Markdown: The definitive
guide}
\end{quote}

The main benefit of R Markdown is readability and reproducibility. Code
written using an R notebook is easy to share, reproduce and understand.
An R Notebook is an R Markdown document with chunks that can be executed
independently and interactively, with output visible immediately beneath
the input. We will be using R Notebooks almost exclusively for this
fellowship.

In addition to using markdown to markup text, R Markdown documents also
use a YAML header. You don't have to learn much about YAML, however, we
need to be familiar enough to perform basic tweaks such as adding a new
title, changing authors etc.

Here are a few references to keep handy

\begin{enumerate}
\def\labelenumi{\arabic{enumi}.}
\tightlist
\item
  \href{https://bookdown.org/yihui/rmarkdown/}{R Markdown: The
  definitive guide}
\item
  \href{https://www.rstudio.com/wp-content/uploads/2016/03/rmarkdown-cheatsheet-2.0.pdf}{R
  Markdown cheatsheet}
\end{enumerate}

\textbf{⚡Ninja Tasks⚡}

\begin{enumerate}
\def\labelenumi{\arabic{enumi}.}
\tightlist
\item
  Create an H2 level header with title Class 1 Exercises
\item
  Write the following text: ``The list below shows the exercises for
  this class''
\item
  Create an unordered list with following list items: ``1. Practice
  Shortcuts'' ``2. Practice dplyr''
\item
  Click on the preview button at the top left of the IDE to preview your
  document.
\end{enumerate}

\subsection{Shortcuts}\label{shortcuts}

Here are a few keyboard shortcuts you should use all the time

\begin{enumerate}
\def\labelenumi{\arabic{enumi}.}
\tightlist
\item
  \texttt{Cmd\ +\ Option\ +\ I} or \texttt{Ctrl\ +\ Alt\ +\ I}: Insert a
  chunk in markdown/notebook etc
\item
  \texttt{Option\ +\ -} or \texttt{Alt\ +\ -}: Enter the assignment
  operator
\item
  \texttt{Cmd\ +\ Enter} or \texttt{Ctrl\ +\ Enter}: Run current
  line/selection
\item
  \texttt{Cmd\ +\ Shift\ +\ Enter} \texttt{Ctrl\ +\ Shift\ +\ Enter}:
  Run the entire script if it is a script file or the current chunk in
  the case of a notebook.
\item
  \texttt{Cmd\ +\ Shift\ +\ M} or \texttt{Ctrl\ +\ Shift\ +\ M}: Insert
  pipe operator
\end{enumerate}

You can find more shortcut keys
\href{https://support.rstudio.com/hc/en-us/articles/200711853-Keyboard-Shortcuts}{here}

\textbf{⚡Ninja Tasks⚡}

\begin{enumerate}
\def\labelenumi{\arabic{enumi}.}
\tightlist
\item
  Use the shortcuts to create a new code chunk \%\textgreater{}\% assign
  the string ``Hello World'' to \texttt{myFirstVar} variable
  \%\textgreater{}\% run the line of code \%\textgreater{}\% add a print
  command using \texttt{print(myFirstVar)} \%\textgreater{}\% run the
  entire chunk
\end{enumerate}

\subsection{Data Manipulation using
dplyr}\label{data-manipulation-using-dplyr}

We will be using the \texttt{nycflights13} package for this section.

\begin{enumerate}
\def\labelenumi{\arabic{enumi}.}
\setcounter{enumi}{1}
\tightlist
\item
  Load the nycflights13 package
\end{enumerate}

\subsection{Post-requisites}\label{post-requisites}

\begin{enumerate}
\def\labelenumi{\arabic{enumi}.}
\tightlist
\item
  Setup a \href{https://github.com}{GitHub Account}
\item
  Install \href{https://desktop.github.com/}{GitHub Desktop}
\item
  Read up about
  \href{https://www.atlassian.com/git/tutorials/what-is-version-control}{version
  control}
\end{enumerate}


\end{document}
